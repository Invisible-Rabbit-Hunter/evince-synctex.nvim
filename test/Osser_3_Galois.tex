\documentclass[a4paper, 11pt]{homework}

\usepackage{common-defs/core}
\usepackage{common-defs/algebra}
\usepackage{common-defs/logic}
\usepackage{amsthm}

\DeclareMathOperator{\cf}{cf}
\DeclareMathOperator{\powerset}{\mathcal{P}}
\DeclareMathOperator{\rank}{rank}
\DeclareMathOperator{\Hom}{Hom}
\DeclareMathOperator{\ord}{ord}
\DeclarePairedDelimiter{\abs}{\vert}{\vert}
\renewcommand{\phi}{\varphi}
\newcommand{\cF}{\mathcal{F}}
\newcommand{\cS}{\mathcal{S}}
\newcommand{\F}{\mathbb{F}}
\newcommand{\bP}{\mathbb{P}}
\newcommand{\Aut}{\mathrm{Aut}}
\newcommand{\Gal}{\mathrm{Gal}}
\newcommand{\gen}[1]{{\providecommand{\given}{}\renewcommand{\given}{\;\vert\;}\langle#1\rangle}}
\newcommand{\divides}{\mathrel{\vert}}
\newcommand{\copi}{\rotatebox[origin=c]{180}{\pi}}
\newtheorem{lemma}{Lemma}
\newcommand{\into}{\hookrightarrow}
\newcommand{\onto}{\twoheadrightarrow}

\author{Jonathan Osser}
\title{Homework 2}

\begin{document}
\maketitle

\begin{questions}
	\question{} Let \(G = \Z/4\Z \times \Z/4\Z \times \Z/12\Z\). Describe explicitly a finite extension of \(\Q\)
	having \(G\) as Galois group.

	\begin{solution}
		Corollary 14.22 of Dummit and Foote yields one way to find a field \(F\) extension whose Galois group is a
		direct product of smaller groups; simply take \(F\) to be the composite of extensions whose pairwise
		intersections are trivial. In this case we would want to have two	distinct fields with Galois group isomorphic
		to \(\Z/4\Z\) such that their intersection is trivial, and one field whose Galois group is isomorphic to
		\(\Z/12\Z\) which has trivial intersection with both other fields.

		An obvious extension with Galois group \(\Z/4\Z\) is the cyclotomic field \(F_1 = \Q(\zeta_5)\) adjoining to
		\(\Q\) a primitive fifth root of unity. By theorem 14.26 of Dummit and Foote the Galois group of the extension
		is isomorphic to \((\Z/5\Z)^\times \cong \Z/4\Z\). The same result gives us an extension with Galois group
		\(\Z/12\Z\), namely \(F_2 = \Q(\zeta_{13})\).

		Finally we use problem 6 of this homework as a hint that \(F_3 = \Q(\sqrt{2+\sqrt{2}})\) has Galois group
		isomorphic to \(\Z/4\Z\), which we will show holds true in that problem.

		Finally we need to show that the intersection of any two of these extensions is trivial. Since both It suffices
		to prove that there is no subfield of degree \(2\) which lies in the intersection of any pair of \(F_1,F_2,F_3
		\). Note that there is only one non-trivial subgroup of \(\Z/4\Z\), so by the fundamental theorem of Galois
		Theory, there is only one non-trivial subfield of \(F_1,F_3\) each, both of degree two. However, \(F_3/\Q(
		\sqrt2)\) and \(F_1/\Q(\sqrt5)\), but \(\Q(\sqrt2) \ne \Q(\sqrt5)\), so \(F_1 \cap F_3 = \emptyset\). For
		\(F_2\), notice that \(\Gal(F_2/\Q) = \Z/12\Z\), which has only one unique subgroup of index \(2\), namely
		the one of order \(6\). This subgroup the corresponds to the subfield \(\Q(\sqrt{13})\), which is neither
		\(\Q(\sqrt2)\) nor \(\Q(\sqrt5)\). Hence the pairwise intersection of the fields \(F_1,F_2,F_3\) is trivial.
		Thus their composite \(F = F_1F_2F_3\) has Galois group \(G\).
	\end{solution}

	\question{} Suppose \(m\) is a positive integer and \(2^m+1\) is prime. Show that \(m\) itself must be a power
	of \(2\).
	\begin{solution}
    Notice that for \(m = 2k+1\) odd we have
		\begin{align*}
			x^m + 1 & = x^{2k+1} + 1                             \\
			        & = x^{2k+1} - x^{2k} + x^{2k} \dots - x + 1 \\
              & = (x^{2k+1} - x^{2k} + \dots - x^2 + x) + (x^{2k} - x^{2k-1} + \dots - x + 1) \\
              & = (x+1)(x^{2k} - x^{2k-1} + \dots - x + 1).
		\end{align*}
    So \(m\) cannot be odd, as then \(2^m+1 = (2+1)(\dots)\) would have a non-trivial a factorisation of \(2^m+1\),
    which contradicts that \(2^m+1\) is prime. In fact, if \(m\) has some non-two factor, i.e. \(m = 2^kn\) for
    \(n\) odd, then
    \[2^m+1 = 2^{2^kn} + 1 = \left(2^{2^k}\right)^n + 1\]
    has a non-trivial factorisation \((2^{2^k}+1)(\dots)\) from the above, which contradicts so \(2^m+1\) being
    prime. Hence \(m\) cannot have a non-two factor, so \(m = 2^k\) for some \(k\).
  \end{solution}

	\question{} Assume that \(m\) is a positive integer, that \(p\) is prime and that \(p\) divides the Fermat
	number \(2^{2^m}+1\). Show that \(p \equiv 1\pmod{2^{m+1}}\).
	\begin{solution}
    Note that \(\Phi_{2^{m+1}}(x) = x^{2^m} + 1\) for \(m \ge 1\).  Since \(p\) divides \(2^{2^m}+1 = \Phi_{2^
    {m+1 }}\) we have that \(\Phi_{2^{m+1}}(2) = 1 \in \F_p\), so \(2\) has order \(2^{m+1}\) in \(\F_p^\times\).
    Hence \(2^{m+1}\) divides \(p-1\), which means \(p \equiv 1 \pmod (2^{m+1})\).
	\end{solution}

  \newpage
	\question{} Let \(K\) be a field. Given \(\alpha_1,\dots,\alpha_n \in K\) with \(n \ge 2\), the
	\textit{Vandermonde determinant} of \(\alpha_1,\dots,\alpha_n\) is
	\[
		V(\alpha_1,\dots,\alpha_n) = \det\begin{pmatrix}
			1      & \alpha_1 & \alpha_1^2 & \cdots & \alpha_1^{n-1} \\
			1      & \alpha_2 & \alpha_2^2 & \cdots & \alpha_2^{n-1} \\
			\vdots & \vdots   & \vdots     & \ddots & \vdots         \\
			1      & \alpha_n & \alpha_n^2 & \cdots & \alpha_n^{n-1}
		\end{pmatrix}.
	\]
	Show that the Vandermonde determinant satisfies \textit{Vandermonde's Identity}:
	\[
		V(\alpha_1,\dots,\alpha_n) = \prod_{j < i}(\alpha_i - \alpha_j).
	\]
	\begin{solution}
		We prove this by induction on \(n\).

		In the base case, when \(n = 2\), we have
		\[
			V(\alpha_1, \alpha_2) = \det\begin{pmatrix}
				1 & \alpha_1 \\
				1 & \alpha_2
			\end{pmatrix} = (1 - \alpha_2)(1 - \alpha_1) = \alpha_2 - \alpha_1,
		\]
		so we are done.

		For the inductive case, assume that for \(k \ge 2\) and any \(\beta_1,\dots,
		\beta_k\) we have \(V(\beta_1,\dots,\beta_k) = \prod_{j < i} (\beta_i - \beta_k)\), i.e. that the identity
		holds for \(n = k\). We need to show it holds for \(n = k + 1\) also. Let \(\alpha_1,\dots,\alpha_{k+1}\)
		be given. Then we can subtract the previous column times \(\alpha_1\) from each but the first column, which
		computes as follows:
		\begin{align*}
			V(\alpha_1,\dots,\alpha_{k+1})
			 & = \det\begin{pmatrix}
				1      & \alpha_1     & \alpha_1^2     & \cdots & \alpha_1^k     \\
				1      & \alpha_2     & \alpha_2^2     & \cdots & \alpha_2^k     \\
				\vdots & \vdots       & \vdots         & \ddots & \vdots         \\
				1      & \alpha_{k+1} & \alpha_{k+1}^2 & \cdots & \alpha_{k+1}^k
			\end{pmatrix} \\
			 & = \det\begin{pmatrix}
				1      & 0                       & 0                                     & \cdots & 0                \\
				1      & \alpha_2 - \alpha_1     & \alpha_2^2 - \alpha_2\alpha_1         & \cdots & \alpha_2^k -
				\alpha_2^{k-1}\alpha_1                                                                               \\
				\vdots & \vdots                  & \vdots                                & \ddots & \vdots           \\
				1      & \alpha_{k+1} - \alpha_1 & \alpha_{k+1}^2 - \alpha_{k+1}\alpha_1 & \cdots & \alpha_{k+1}^k -
				\alpha_{k+1}^k \alpha_1^k
			\end{pmatrix} \\
			 & = \det\begin{pmatrix}
				1      & 0                       & 0                                     & \cdots & 0                   \\
				1      & \alpha_2 - \alpha_1     & \alpha_2(\alpha_2 - \alpha_1)         & \cdots & \alpha_2^{k-1}
				(\alpha_2 -	\alpha_1)                                                                                    \\
				\vdots & \vdots                  & \vdots                                & \ddots & \vdots              \\
				1      & \alpha_{k+1} - \alpha_1 & \alpha_{k+1}(\alpha_{k+1} - \alpha_1) & \cdots & \alpha_{k+1}^{k-1}(
				(\alpha_{k+1} - \alpha_1)                                                                               \\
			\end{pmatrix} \\
			 & = \det\begin{pmatrix}
				\alpha_2 - \alpha_1     & \alpha_2(\alpha_2 - \alpha_1)         & \cdots & \alpha_2^{k-1}
				(\alpha_2 - \alpha_1)                                                                          \\
				\vdots                  & \vdots                                & \ddots & \vdots              \\
				\alpha_{k+1} - \alpha_1 & \alpha_{k+1}(\alpha_{k+1} - \alpha_1) & \cdots & \alpha_{k+1}^{k-1}(
				(\alpha_{k+1} - \alpha_1)                                                                      \\
			\end{pmatrix} \\
			 & =
			\left(\prod_{2 \le i} (\alpha_i - \alpha_1)\right) \det\begin{pmatrix}
				1      & \alpha_2 + f_2(\alpha_2)         & \cdots & \alpha_2^{k-1} + f_k(\alpha_2)         \\
				\vdots & \vdots                           & \ddots & \vdots                                 \\
				1      & \alpha_{k+1} - f_2(\alpha_{k+1}) & \cdots & \alpha_{k+1}^{k-1} + f_k(\alpha_{k+1})
			\end{pmatrix}.
		\end{align*}
		The latter determinant is \(\prod_{2 \le j < i} (\alpha_i - \alpha_j)\) by the inductive hypothesis,
		so combined with the factor in front we have
		\[
			V(\alpha_1,\dots,\alpha_{k+1}) = \left(\prod_{2\le i}(\alpha_i - \alpha_1)\right)\left
			(\prod_{2 \le j < i} (\alpha_i - \alpha_j)\right) = \prod_{j < i}(\alpha_i - \alpha_j),
		\]
		which is what we aimed to prove.
		Hence we conclude by induction that the statement holds for all \(n \ge 2\).
	\end{solution}


	\question{} Let \(K = \Q(\sqrt[8]{2},i)/\Q\). Consider its subfields \(F_1 = \Q(i)\), \(F_2 = \Q(\sqrt2)\), and
	\(F_3 = \Q(\sqrt{-2})\). Show that \(\Gal(K/F_1) \cong \Z/8\Z\), that \(\Gal(K/F_2)\) is dihedral of order 8, and
	that \(\Gal(K/F_3)\) is isomorphic to the quaternion group of order \(8\).
	\begin{solution}

		Let \(\theta = \sqrt[8]{2}\) and \(\zeta\) be a primite 8th root of unity, as in the example on page 577 of
		Dummit and Foote. Then we have from said example that
		\[\Gal(K/\Q) = \gen{\sigma,\tau \given \sigma^8 = \tau^2 = 1, \sigma\tau = \tau\sigma^3},\]
		where
		\[
			\sigma : \begin{cases}
				\theta \mapsto \zeta\theta \\
				i \mapsto i
			\end{cases}
			\text{\quad and \quad}
			\tau : \begin{cases}
				\theta \mapsto \theta \\
				i \mapsto -i.
			\end{cases}
		\]
		This group has order 16. The order of each of the Galois groups \(\Gal(K/F_i), 1 \le i \le 3\) is 8,
		since the index of each subgroup in \(\Gal(K/\Q)\) is 2, which follows from the fact that each field is a
		quadratric extension of \(\Q\), and thus of degree 2.

		Clearly the maps which fix \(i\) are precisely the powers of \(\sigma\), so from this we conclude \(\Gal(K/
		F_1) = \gen{\sigma}\), and since \(\sigma\) has order 8, we have \(\gen{\sigma} \cong \Z/8\Z\) by definition.
		Hence \(\Gal(K/F_1) \cong \Z/8\Z\).

		Similarly, the maps which fix \(\sqrt 2 = \theta^4\) are \(\tau\), as well as not \(\sigma\), since
		\(\sigma(\theta^4) = \sigma(\theta)^4 = (\zeta\theta)^4 = -\theta^4\), but rather the powers of \(\sigma^2\):
		\[
			\sigma^2(\theta^4) = \sigma(\sigma(\theta^4)) = \sigma(-\theta^4) = -\sigma(\theta^4) = \theta^4.
		\]
		Clearly \(\tau \in \Gal(K/F_2)\). Note that \(\gen{\sigma^2,\tau}\) has order \(8\) and is a subgroup of
		\(\Gal(K/F_2)\), so \(\Gal(K/F_2) = \gen{\sigma^2,\tau}\). This group satisfies the equation \(\sigma^2\tau =
		\sigma\tau\sigma^3 = \tau\sigma^6 = \tau\sigma^{-2}\), i.e. is dihedreal. Thus \(\Gal(K/F_2)\) is dihedral of
		order 8.

		Finally we wish to find the automorphisms which fix \(\sqrt{-2} = \sqrt2 i = \theta^4 i\). Note that
		\(\sigma^2(\theta^4i) = \theta^4i\) and \(\sigma\tau(\theta^4i) = \sigma\tau(\theta^4)\sigma\tau(i) =
		(-\theta^4)(-i) = \theta^4i\), so \[\sigma^2, \sigma\tau \in \Gal(K/F_3).\] Furthermore any product of
		those are in \(\Gal(K/F_3)\), so \(\gen{\sigma^2, \sigma\tau} \le \Gal(K/F_3)\). Note that
		\[
			\gen{\sigma^2,\sigma\tau} = \Set{1, \sigma^2, \sigma^4, \sigma^6, \sigma\tau, \sigma^3\tau, \sigma^5\tau
				\sigma^7\tau}
		\]
		has order 8. Hence \(\Gal(K/F_3) = \gen{\sigma^2, \sigma\tau}\). This group has at least three elements whose
		square is \(\sigma^4\),  namely \(\sigma^2\), \(\sigma\tau\), \(\tau\sigma\):
		\begin{align*}
			(\sigma^2)^2   & = \sigma^4                                                 \\
			(\sigma\tau)^2 & = \sigma\tau\sigma\tau = \sigma\tau\tau\sigma^3 = \sigma^4 \\
			(\tau\sigma)^2 & = \tau\sigma\tau\sigma = \tau\tau\sigma^4 = \sigma^4.
		\end{align*}
		Since the group is non-abelian (\(\tau\sigma \sigma^2 = \tau\sigma^3 = \sigma\tau \ne \sigma^2 \sigma\tau\))
		and has more than one element of order 4, it cannot be the dihedral group, and so must be the quaternion group.
	\end{solution}

	\question{} Show that \(\Q(\sqrt{2+\sqrt2})/\Q\) is Galois with Galois group \(\Z/4\Z\).
	\begin{solution}
		The minimal polynomial for \(\alpha = \sqrt{2+\sqrt2}\) is \(m_\alpha = x^4-4x^2+2\), whose other roots are
		\(\beta = \sqrt{2 - \sqrt2}\), \(-\alpha,\) and \(-\beta\), or rather, the roots of \(m_\alpha\) are \(\Set{
			\pm\alpha,\pm\beta}\). By computation we can show that \(\beta = \frac{\sqrt2}\alpha\), since
		\[
			\alpha\beta = \sqrt{(2+\sqrt2)(2-\sqrt2)} = \sqrt2 = \alpha^2 - 2 = 2 - \alpha^2,
		\]
		so \(\beta = \frac{\alpha\beta}{\alpha} = \frac{\alpha^2-2}{\alpha}\) and \(\alpha = \frac{2-\beta^2}{\beta}\).

		The minimal polynomial is irreducible over \(\Q\), so the Galois group acts transitively on the roots. So
		we prove by cases that there Galois groups is isomorphic to \(\Z/4\Z\):
		\begin{itemize}
			\item If \(\sigma(\alpha) = \alpha\), then \(\sigma(\beta) = \frac{\sigma(\alpha)^2-2}{\sigma(\alpha)}
			      = \beta\), so \(\sigma\) fixes the whole extension, i.e. \(\sigma\) is the identity map.
			\item If \(\sigma(\alpha) = \beta\), then \(
			      \sigma(\beta) = \frac{\sigma(\alpha)^2-2}{\sigma(\alpha)}
			      = \frac{\beta^2-2}{\beta} = -\alpha\).
			\item If \(\sigma(\alpha) = -\alpha\), then \(
			      \sigma(\beta) = \frac{\sigma(\alpha)^2-2}{\sigma(\alpha)}
			      = \frac{\alpha^2-2}{-\alpha} = -\beta\).
			\item If \(\sigma(\alpha) = -\beta\), then \(
			      \sigma(\beta) = \frac{\sigma(\alpha)^2-2}{\sigma(\alpha}
			      = \frac{\beta^2 - 2}{-\beta} = \alpha\).
		\end{itemize}
		Since any permutation is uniquely determined by where it sends \(\alpha\), the order of the Galois group
		must be 4, and furthermore any automorphism is determined by the automorphism which satisifies \(\sigma(
		\alpha) = \beta\): \(\sigma^2(\alpha) = -\alpha\), \(\sigma^3(\alpha) = -\beta\), and \(\sigma^4(\alpha) =
		\alpha\), so \(\sigma^4 = \id\), whence \(\Gal(\Q(\alpha)/\Q) = \Set{\id, \sigma, \sigma^2, \sigma^3}\).
	\end{solution}

	\question{} Let \(p_1,\dots,p_n\) be \(n\) distinct primes. Show that \(\Q(\sqrt{p_1},\dots,\sqrt{p_n})/\Q\) is
	Galois with Galois group isomorphic to \((\Z/2\Z)^n\).
	\begin{solution}
		Clearly \(\Q(\sqrt{p_1},\dots,\sqrt{p_n})\) is the least field extension over \(\Q\) for which the polynomial
		\(f(x) = \prod_{k=1}^n (x^2-p_k)\) splits completely, and each factor is an irreducible polynomial of degree
		2. Hence no factor has a multiple root (Corollary 13.34 of Dummit and Foote), so \(f\) is separable. The
		Galois group of each factor is \(\Z/2\Z\), so the Galois group of \(f\) is \((\Z/2\Z)^n\). Therefore
		\(\Gal(\Q(\sqrt{p_1}\dots,\sqrt{p_n})/\Q) = (\Z/2\Z)^n\).
	\end{solution}

	\question{} Determine the Galois group of \(x^5 + x - 1\).
	\begin{solution}
		We factor the polynomial \(x^5 + x - 1 = (x^2 - x + 1)(x^3 + x^2 - 1)\). Clearly the quadratic factor \(p(x)\)
		is irreducible: it only has complex roots \(x = \frac{1}{2} \pm \frac{\sqrt{-3}}{2}\). Similarly the cubic
		\(q(x)\) is also irreducible. The discriminant of \(q(x)\) is \(-23\), which is not square in \(\Q\), so the
		Galois group of \(q(x)\) is \(S_3\). Similarly the Galois group of \(p(x)\) is \(S_2 \cong \Z/2\Z\). So the
		Galois group of \(x^5 + x - 1\) is \(\Gal(E_pE_q/\Q)\), where \(E_p, E_q\) are the splitting field of \(p(x)\)
		and \(q(x)\) respectively. Note that \(E_p = \Q(\sqrt{-3})\) and \(\Q(\sqrt{-23}) \subset E_q\). Furthermore
		\(E_q\) has precisely one quadratic subfield, the fixed field of \(A_3\), since \(A_3\) is the only subgroup
		of \(S_3\) with index 2. So if \(E_p\) and \(E_q\) had non-trivial intersection, then \(E_p \subseteq E_q\),
		so \(\Q(\sqrt{-3}) = E_p = \Q(\sqrt{-23})\), which is obviously false. Hence \(E_p \cap E_q = \Q\), from which
		follows that
		\[\Gal(E_pE_q/\Q) = \Gal(E_p/\Q) \times \Gal(E_q/\Q) = S_3 \times \Z/2\Z,\]
		which is thus the Galois field of the polynomial \(x^5 - x + 1\) over \(\Q\).
	\end{solution}

	\question{} Let \(p\) be a prime. Assume that \(x^5 + ax + b \in \F_p[x]\) is irreducible over \(\F_p\). Show
	that \(5^5b^4 + 4^4a^5\) is a square in \(\F_p^\times\).
	\begin{solution}
		Since \(x^5 + ax + b\) is irreducible, its splitting field is of degree 5 over \(\F_p\). Hence its Galois
		group is some cyclic subgroup \(C_5\) of \(S_5\) consisting of \(5\)-cycles and identity. Since every element
		in \(\C_5\) is either identity or a 5-cycle, all elements are even, so \(C_5 \le A_5\). Hence it follows from
		proposition 34 that the discriminant is a square in \(F\). But the discriminant is \(5^5b^4+4^4a^5\), so
		\(5^5b^4 + 4^4a^5 \in \F_p\) is square. Furthermore the discriminant is zero precisely if there are multiple
		roots, but its irreducible and thus seperable. Hence \(5^5b^4 + 4^4a^5\) is non-zero, so its in \(\F_p^\times
		\).
	\end{solution}

	\question{} Let \(p\) be a prime. Show that the extension \(\F_p(x,y)/\F_p(x^p,y^p)\) does not admit a primitive
	element.
	\begin{solution}
		Let \(F = \F_p(x^p,y^p)\). Then we have a family of extensions \(F(x+y^{kp+1})\), which are obviously subfields
		of \(\F_p(x,y)\) and each are of degree \(p\). Furthermore no two fields in this family of different values
		\(k\) are equal. For \(n \ne m\), if \(F(x+y^{np+1}) = F(x+y^{mp+1})\), then
		\[(x+y^{np+1}) - (x + y^{mp+1}) = y^{np+1} - y^{mp+1} = y(y^{np} - y^{mp}),\]
		but \(y^{np} - y^{mp} \ne 0\), so \(y \in F(x+y^{np+1})\), from which follows that \(x \in F(x+y^{np+1})\), so
		then \(F(x+y^{np+1}) = \F_p(x,y)\). However, by each field in the family is of degree \(p\), since
		\((x+y^{kp+1})^p = x^p+y^{kp^2+p} \in \F_p(x^p,y^p\), whereas \(\F_p(x,y)\) is of degree \(p^2\) over
		\(\F_p(x^p, y^p)\). Hence all \(F(x+y^{kp+1}\) are distinct, and there are infinitely many, so by proposition
		14.24 of Dummit and Foote it follows that \(K\) is not a simple extension, so it admits no primitive element.
	\end{solution}
\end{questions}

\end{document}
